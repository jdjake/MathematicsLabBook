%%% Laboratory   Notes
%%% Template by Mikhail Klassen, April 2013
%%% Contributions from Sarah Mount, May 2014
\documentclass[openany,nobib,nols,a4paper,twoside,symmetric,justified,notoc]{tufte-book}

\newcommand{\userName}{Jacob Denson}

%%%%%%%%%%%%%%%%%%%%%%%%%%%%%%%%%%%%%%%
% Generic Packages
%%%%%%%%%%%%%%%%%%%%%%%%%%%%%%%%%%%%%%%

\usepackage{natbib}
\usepackage{comment}
\usepackage{amsmath}
\usepackage{amssymb}

%%%%%%%%%%%%%%%%%%%%%%%%%%%%%%%%%%%%%%%
% Theorem Specification
%%%%%%%%%%%%%%%%%%%%%%%%%%%%%%%%%%%%%%%

\usepackage{amsthm}

\theoremstyle{plain}
\newtheorem{theorem}{Theorem}[chapter]
\newtheorem{axiom}{Axiom}
\newtheorem{lemma}[theorem]{Lemma}
\newtheorem{corollary}[theorem]{Corollary}
\newtheorem{prop}[theorem]{Proposition}
\newtheorem{exercise}{Exercise}[chapter]
\newtheorem{fact}{Fact}[chapter]

\newtheorem*{example}{Example}
\newtheorem*{proof*}{Proof}

\theoremstyle{remark}
\newtheorem*{exposition}{Exposition}
\newtheorem*{remark}{Remark}
\newtheorem*{remarks}{Remarks}

\theoremstyle{definition}
\newtheorem*{defi}{Definition}

%%%%%%%%%%%%%%%%%%%%%%%%%%%%%%%%%%%%%%%
% Header / Footer Specification
%%%%%%%%%%%%%%%%%%%%%%%%%%%%%%%%%%%%%%%

\lhead{\textsc{\userName}}
\chead{\textsc{Lab Notes}}
\cfoot{~~\textit{Last modified: \today}}
\rfoot{\textsc{\thepage}}

%%%%%%%%%%%%%%%%%%%%%%%%%%%%%%%%%%%%%%%
% Environments
%%%%%%%%%%%%%%%%%%%%%%%%%%%%%%%%%%%%%%%

\newenvironment{importanttodo}%
    {\subsection{Important To Do List}}%
    {\vspace{2mm}\hrule\hspace{\stretch{1}}\\}

\newenvironment{maybedo}%
    {\subsection{Maybe Do}}%
    {\vspace{2mm}\hrule\hspace{\stretch{1}}\\}

%%%%%%%%%%%%%%%%%%%%%%%%%%%%%%%%%%%%%%%
% Table of Contents
%%%%%%%%%%%%%%%%%%%%%%%%%%%%%%%%%%%%%%%
\usepackage{lipsum,mdframed}
\definecolor{secnum}{RGB}{13,151,225}
\definecolor{ptcbackground}{RGB}{255,255,255}
\definecolor{ptctitle}{RGB}{0,177,235}
\usepackage{titletoc}
\usepackage{etoolbox}
\setcounter{tocdepth}{1}
\pretocmd{\tableofcontents}{\begin{mdframed}[backgroundcolor=ptcbackground,outermargin=\dimexpr-\marginparwidth-\marginparsep\relax,innermargin=0pt,hidealllines=true]\let\cleardoublepage\relax}{}{}
\apptocmd{\tableofcontents}{\end{mdframed}}{}{}


%%%%%%%%%%%%%%%%%%%%%%%%%%%%%%%%%%%%%%%
% COLORS (Tango) Mostly by Philip Bunge
% http://pbunge.crimson.ch/
%%%%%%%%%%%%%%%%%%%%%%%%%%%%%%%%%%%%%%%
\definecolor{White}{gray}{0.9}
\definecolor{Black}{gray}{0.0}
\definecolor{LightButter}{rgb}{0.98,0.91,0.31}
\definecolor{LightOrange}{rgb}{0.98,0.68,0.24}
\definecolor{LightChocolate}{rgb}{0.91,0.72,0.43}
\definecolor{LightChameleon}{rgb}{0.54,0.88,0.20}
\definecolor{LightSkyBlue}{rgb}{0.45,0.62,0.81}
\definecolor{LightPlum}{rgb}{0.68,0.50,0.66}
\definecolor{LightScarletRed}{rgb}{0.93,0.16,0.16}
\definecolor{Butter}{rgb}{0.93,0.86,0.25}
\definecolor{Orange}{rgb}{0.96,0.47,0.00}
\definecolor{Chocolate}{rgb}{0.75,0.49,0.07}
\definecolor{Chameleon}{rgb}{0.45,0.82,0.09}
\definecolor{SkyBlue}{rgb}{0.20,0.39,0.64}
\definecolor{Plum}{rgb}{0.46,0.31,0.48}
\definecolor{ScarletRed}{rgb}{0.80,0.00,0.00}
\definecolor{DarkButter}{rgb}{0.77,0.62,0.00}
\definecolor{DarkOrange}{rgb}{0.80,0.36,0.00}
\definecolor{DarkChocolate}{rgb}{0.56,0.35,0.01}
\definecolor{DarkChameleon}{rgb}{0.30,0.60,0.02}
\definecolor{DarkSkyBlue}{rgb}{0.12,0.29,0.53}
\definecolor{DarkPlum}{rgb}{0.36,0.21,0.40}
\definecolor{DarkScarletRed}{rgb}{0.64,0.00,0.00}
\definecolor{Aluminium1}{rgb}{0.93,0.93,0.92}
\definecolor{Aluminium2}{rgb}{0.82,0.84,0.81}
\definecolor{Aluminium3}{rgb}{0.73,0.74,0.71}
\definecolor{Aluminium4}{rgb}{0.53,0.54,0.52}
\definecolor{Aluminium5}{rgb}{0.33,0.34,0.32}
\definecolor{Aluminium6}{rgb}{0.18,0.20,0.21}

%%%%%%%%%%%%%%%%%%%%%%%%%%%%%%%%%%%%%%%
% Hyperrefs
%%%%%%%%%%%%%%%%%%%%%%%%%%%%%%%%%%%%%%%

\usepackage{hyperref}
\hypersetup{
    pdffitwindow=false,            % window fit to page
    pdfstartview={Fit},            % fits width of page to window
    pdftitle={Lab notes 2014},     % document title
    pdfauthor={Your Name},         % author name
    pdfsubject={},                 % document topic(s)
    pdfnewwindow=true,             % links in new window
    colorlinks=true,               % coloured links, not boxed
    linkcolor=DarkScarletRed,      % colour of internal links
    citecolor=DarkChameleon,       % colour of links to bibliography
    filecolor=DarkPlum,            % colour of file links
    urlcolor=DarkSkyBlue           % colour of external links
}

%%%%%%%%%%%%%%%%%%%%%%%%%%%%%%%%%%%%%%%
% Custom Commands
%%%%%%%%%%%%%%%%%%%%%%%%%%%%%%%%%%%%%%%

\DeclareMathOperator{\TT}{\mathbf{T}}
\DeclareMathOperator{\ZZ}{\mathbf{Z}}
\DeclareMathOperator{\RR}{\mathbf{R}}
\DeclareMathOperator{\CC}{\mathbf{C}}

\DeclareMathOperator{\fordim}{\text{dim}_{\mathbf{F}}}
\DeclareMathOperator{\minkdim}{\text{dim}_{\mathbf{M}}}
\DeclareMathOperator{\hausdim}{\text{dim}_{\mathbf{H}}}
\DeclareMathOperator{\modfordim}{\text{dim}_{\mathbf{MF}}}
\DeclareMathOperator{\lowlocdim}{\underline{\text{dim}_{\mathbf{H}}}}
\DeclareMathOperator{\uplocdim}{\overline{\text{dim}_{\mathbf{H}}}}

\title{Laboratory Journal}
\author{Jacob Denson}

\begin{document}

\maketitle
\tableofcontents
\newpage










\chapter{Saturday, 30 May 2020} \label{30052020}

\begin{importanttodo}
\begin{itemize}
    \item Formally prove the space $X$ given by the norms in \eqref{OurFrechetSpace} is complete. \textbf{See \nameref{01062020}}

    \item Try and understand the probabalistic smoothness calculations given in Lemma 7.4 of \cite{Korner2009}. Once this is done, we can try adapting it to our more general situation.
\end{itemize}
\end{importanttodo}

\begin{maybedo}
\begin{itemize}
    \item Today I found an interesting survey on the Fourier Dimension \cite{Ekstrom2017}. If I find the time I should read through it more thoroughly to get some intuition. \textbf{See \nameref{31052020}}

    \item I also found a survey on the application of the probabilistic method and the Baire category theorem in Harmonic analysis \cite{Kahane2000}. I feel this method is very exploitable in the types of problems I currently deal with, so if only for culture, this should be a useful read.
\end{itemize}
\end{maybedo}

My main goals today were to finish up the slides, for my talk on Fourier dimension in pattern avoidance problems at the 2020 Ottawa Math Conference. I have essentially completed these slides; all that remains is to polish them up, and practice giving the presentation. My main goal during the presentation is to show that viewing avoiding sets $Z$ geometrically leads to interesting questions, that the geometric quantities we consider lead to important consequences, and that the Fourier dimension question I am currently considering is interesting to study. I also thought about an idea which seemed to prevent adapting our result obtained by the `queuing approach' in \cite{Denson2019} to Thomas K\"{o}rner's Baire category approach in \cite{Korner2009}, as well as finding a counterexample in a paper which removes the trivial case of \cite{Denson2019} from the Fourier dimension case.

%-----------------------------------------

\section{Ideas for Fourier Dimension Technique} \label{IdeasForFourierDimensionTechnique}

Thomas K\"{o}rner's paper \cite{Korner2009} relies on Baire category arguments to construct generic measures $\mu$ supported on a subset of $\TT$ avoiding solutions to $m$-term linear equations, such that for each $\xi \in \ZZ$,
%
\begin{equation} \label{Korner2009Bound}
    |\widehat{\mu}(\xi)| \leq A(\xi).\footnotemark
\end{equation}
%
\footnotetext{Here $\{ A(\xi) \}$ is a sequence given for each $\xi \in \ZZ$ by the formula
%
\[ A(\xi) = B(\xi) |\xi|^{-\beta/2} \log(1 + |\xi|)^{1/2}, \]
%
where $\beta = (n-1)^{-1}$, and $\{ B(\xi) \}$ is some fixed sequence of positive numbers such that $B(\xi) \to \infty$ as $|\xi| \to \infty$.}%
%
To obtain this bound generically, K\"{o}rner works in the norm space $X$ consisting of finite measures $\mu$ on $\TT$ such that the quantity
%
\begin{equation} \label{KornerBanachSpace}
    \| \mu \|_X = \sup_{\xi \in \ZZ} \frac{|\widehat{\mu}(\xi)|}{A(\xi)}.
\end{equation}
%
is finite. Then any measure in $X$ satisfies \eqref{Korner2009Bound} up to a multiplicative constant, and $X$ is a Banach space, which enables one to use Baire-category techniques. A problem occurs in our Fourier dimension paper because I believe we can only construct finite measures $\mu$ such that for each $\varepsilon > 0$,
%
\begin{equation} \label{OurFourierDimensionBound}
    |\widehat{\mu}(\xi)| \lesssim_\varepsilon |\xi|^{\varepsilon-\beta/2}.
\end{equation}
%
Such a measure does not satisfy quite as rigid an inequality as \eqref{Korner2009Bound}, instead having to satisfy infinitely many inequalities of the form \eqref{OurFourierDimensionBound}, and as such I do not believe we can find a Banach space norm which encapsulates \eqref{OurFourierDimensionBound}.

However, today I thought of an idea which might prove fruitful. For any measure $\mu$ satisfying $\eqref{OurFourierDimensionBound}$ for each $\varepsilon > 0$, the quantities
%
\begin{equation} \label{OurFrechetSpace}
    \| \mu \|_\varepsilon = \sup_{\xi \in \ZZ} |\widehat{\mu}(\xi)| |\xi|^{\beta/2-\varepsilon}
\end{equation}
%
will be finite for all $\varepsilon > 0$. If we let $X$ denote the family of all finite measures which satisfy \eqref{OurFrechetSpace} for all $\varepsilon > 0$, then the collection of seminorms $\{ \| \cdot \|_\varepsilon : \varepsilon > 0 \}$ might give $X$ the structure of a Frech\'{e}t space. Since Frech\'{e}t spaces are complete metric spaces, we can still apply Baire category arguments here. It remains to check whether this really gives a complete metric space structure, however.

\section{Finite Additivity of the Fourier Dimension}

In \cite{Ekstrom2014}, I found a result which constructs two disjoint, Borel sets $A, B \subset \TT$, with $\fordim(A), \fordim(B) < 1$, but such that $A \cup B = \TT$. The result of Theorem 1 of \cite{Denson2019} is trivial when $\minkdim(Z) < d$, for if $\pi: (\TT^d)^n \to \TT^d$ is given by projection onto the first $d$ coordinates, then $\TT^d - \pi(Z)$ has full Hausdorf dimension and avoids $Z$. If $Z = A \times \{ 0 \}$, then $\TT - \pi(Z) = B$, which is not full dimensional, so things are more complicated when dealing with Fourier dimension. I found this result in \cite{Ekstrom2017}, which might be a useful survey to read through completely in order to get a better grasp on how the Fourier dimension behaves.










\chapter{Sunday, 31st May 2020} \label{31052020}

\begin{maybedo}
\begin{itemize}
    \item Read Kahane's book \cite{Kahane1994} to obtain ideas on possible Fourier dimension constructions.

    \item Read Lyon's article \cite{Lyons1995} to obtain cultural background information about Fourier dimension in harmonic analysis.
\end{itemize}
\end{maybedo}

Today I finished off my slides, prepared for the talk, and review Ekstr\"{o}m and Schmeling's survey on Fourier dimension results \cite{Ekstrom2017}. The review is given below.

\section{Ekstr\"{o}m and Schmeling's ``A Survey of Fourier Dimension''}

This article gives a pretty diverse set of viewpoints about the Fourier dimension. I'll list off a couple of things that I hadn't heard about before:
%
\begin{itemize}
    \item It is obvious that if $\mu$ is a finite Borel measure with Hausdorff dimension $s$ and $\mu(E) > 0$, then $\hausdim(E) \geq s$. Thus $\mu$ does not need to be supported on $E$ completely. However, if we define the \emph{modified Fourier dimension}
    %
    \begin{equation} \label{ModifiedFourierDimension}
        \modfordim(E) = \sup \{ \fordim(\mu) : \mu(E) > 0 \},
    \end{equation}
    %
    we get a different dimension to the Fourier dimension
    %
    \begin{equation} \label{UsualFourierDimension}
        \fordim(E) = \sup \{ \fordim(\mu) : \text{supp}(\mu) \subset E \}.
    \end{equation}
    %
    The modified Fourier dimension \emph{is} countably stable, i.e. for a countable collection of Borel sets $\{ E_k \}$,
    %
    \[ \modfordim \left( \bigcup_k E_k \right) = \sup_k \left( \fordim(E_k) \right) \]
    %
    The usual Fourier dimension is not even finitely stable, but is countably stable if we restrict ourselves to $F_\delta$ sets.

    \item It is interesting that we can define the Hausdorff dimension completely locally. Given a finite Borel measure $\mu$ on $\RR^d$, and $x \in \RR^d$, define the \emph{local dimension} as
    %
    \[ \lowlocdim(\mu,x) = \liminf_{r \to 0} \log_r \mu(B_x(r)). \]
    %
    and
    %
    \[ \uplocdim(\mu,x) = \limsup_{r \to 0} \log_r \mu(B_x(r)). \]
    %
    \begin{theorem}
        If $\mu(E) > 0$ and $\lowlocdim(\mu,x) \geq s$ for $\mu$ a.e. $x \in E$, then $\hausdim(E) \geq s$.
    \end{theorem}
    \begin{proof}
        For each $\varepsilon > 0$, there exists $r_0 > 0$ and $F \subset E$ with $\mu(F) > 0$ such that for each $x \in F$ and $r \leq r_0$, $\mu(B_x(r)) \geq r^{s - \varepsilon}$. This implies $\hausdim(E) \geq \hausdim(F) \geq s - \varepsilon$. We then take $\varepsilon \to 0$ to conclude $\hausdim(E) \geq s$.
    \end{proof}
    % 
    Working locally might simplify calculations, rather than having to come up with uniform bounds on the dimension. There is no local dimension for the usual Fourier transform, since the Fourier transform is not monotone on sets. However, the modified Fourier dimension \emph{is} monotone, so it might be possible to find a local charactierization of the Fourier dimension. For a measure $\mu$, we define
    %
    \[ \modfordim(\mu) = \sup \{ \fordim(\eta): \mu \ll \eta \}, \]
    %
    so that $\modfordim(E) = \sup \{ \modfordim(\mu): \text{supp}(\mu) \subset E \}$. Then we can try and define the Fourier dimension locally.

    \item Given a family of measures $M$ on a measure space $X$, it is interesting to find a family $\mathcal{U}$ of measurable subsets of $X$ such that a measure $\mu$ on $X$ is an element of $M$ if and only if $\mu(E) = 0$ for all $E \in \mathcal{U}$. For instance, this is possible if a measure $\mu$ is fixed, and $M$ is the family of all absolutely continuous measures with respect to $\mu$. This is also possible for measures with Hausdorff dimension exceeding some value $s$ - it suffices to consider the family of all sets with $s$ Hausdorff measure zero. The family of \emph{Rajchman measures} on $\RR$ are the collection of all measures $\mu$ such that
    %
    \[ \lim_{|\xi| \to \infty} \widehat{\mu}(\xi) = 0. \]
    %
    Suprisingly, one can find a family of sets $\mathcal{U}$ such that a measure $\mu$ is Rajchman if and only if it assigns mass zero to each set $E \in \mathcal{U}$. This is detailed in \cite{Lyons1995}, which might also be a useful text to read for culture. If we let
    %
    \[ M^\perp = \{ E \subset X: \mu(E) = 0\ \text{for all $\mu \in M(X)$} \} \]
    %
    and
    %
    \[ M^{\perp \perp} = \{ \mu : \mu(E) = 0\ \text{for all $E \in M^\perp$} \}. \]
    %
    To show such a family exists for a given collection of measures $M$, it is necessary and sufficient to show $M^{\perp \perp} = M$. This is true for the family of measures with \emph{modified Fourier dimension} greater than a given value, but not for the family of measures whose normal Fourier dimension is greater than a given value. It is an open question to explicitly identify the family of sets in these examples, however, which might be an interesting problem to work on.

    \item If $E$ is a Borel subset of $\RR$, then there exists a $C^{m + \alpha}$ diffeomorphism $f: \RR \to \RR$ such that
    %
    \[ \fordim(f(E)) \geq \frac{\hausdim(E)}{m + \alpha}, \]
    %
    In particular, if $m = 1$ and $\alpha = 0$, any Borel set is diffeomorphic to a Salem set. One can thus view the difference between Hausdorff and Fourier dimension as a measure of pertubation, in some sense. I should check whether it is possible to find a smooth faimly of deformations of $E$ than smoothly deform the Fourier dimension up to the Hausdorff dimension. It is also an open question to construct an \emph{explicit} diffeomorphism $f: \RR \to \RR$ between the Cantor set and a Salem set, since the construction above is obtained in a random fashion.
\end{itemize}

Looking at the references shows a list of books that might be useful to consult for intuition on the problems I'm working on now:
%
\begin{itemize}
    \item J.P. Kahane's book \cite{Kahane1994} on random functions might be useful to obtain ideas on how to obtain Fourier dimension measures.

    \item R. Lyons article \cite{Lyons1995} might be useful for learning some cultural information about the history of Fourier dimension in harmonic analysis.
\end{itemize}





\chapter{Monday 1st June 2020} \label{01062020}

\begin{importanttodo}
\begin{itemize}
    \item Edit the proof in our existing draft on Fourier dimension to work completely rigorously for the new Frech\'{e}t space. \textbf{See \nameref{02062020}}
\end{itemize}
\end{importanttodo}

I gave my presentation at the Ottawa Math Conference today. It went fairly well, I got a few questions so some people followed the talk. Later on in the day I wrote a formal proof that the space I thought up in \nameref{30052020} is actually a Frech\'{e}t space. What remains is to modify the existing proof in the paper draft to completely rigorously prove that generic sets in the space are Salem and avoid patterns.







\chapter{Tuesday 2nd June 2020} \label{01062020}

Today was mostly a writing day. I finished editing the proof in our paper to incorporate the new Frech\'{e}t space idea, so now we can find sets with Fourier dimension
%
\[ \frac{nd - \alpha}{n - 1/2} \]
%
avoiding patterns. All that remains to complete this project is to understand K\"{o}rner's probabilistic analysis which enabled him to improve this bound in the special case of integer coefficient equations, see if we can generalize this to more general cases to yield a bound
%
\[ \frac{nd - \alpha}{n - 1}, \]
%
and then incorporate this into the paper.






\begin{comment}

%%%%%%%%%%%%%%%%%%%%%%%%%%%%%%%%%%%%%%%%%%%%%%%%%%%%%%%%

\newday{10 June 2014}

The (very fast) OCaml to Javascript compiler described in \citep{VouillonBalat13}\footnote{\url{http://ocsigen.org/js_of_ocaml/}} takes the unusual approach of compiling OCaml \textit{bytecode} to Javascript, rather than performing a source-to-source translation.  

\hrulefill

%%%%%%%%%%%%%%%%%%%%%%%%%%%%%%%%%%%%%%%%%%%%%%%%%%%%%%%%

\newday{9 June 2014}

Poor benchmark results. Ideas for improvement are listed in the issue tracker in the repository.

%\includegraphics[scale=0.65]{benchmark}

\hrulefill

%%%%%%%%%%%%%%%%%%%%%%%%%%%%%%%%%%%%%%%%%%%%%%%%%%%%%%%%

\newday{6 June 2014}

Back from vacation.

\newthought{Student project idea} improve \url{https://github.com/snim2/Terminus} by adding new BASH commands.

Size of data sets for the ngram paper are shown in Table \ref{tab:ngram-sizes}.

\begin{table}[p]
\label{tab:ngram-sizes}
\centering
\caption{Data sizes in Google ngram data} 
\begin{tabular}{lrr} 
\toprule
Data    &   Rows &  Compressed Size (GB)\\
\midrule
English\\
\midrule
1 gram &    472,764,897 &   4.8\\
2 gram &    6,626,604,215 & 65.6\\
3 gram &    23,260,642,968 &    218.1\\
4 gram &    32,262,967,656 &    293.5\\
5 gram &    24,492,478,978 &    221.5\\
\textbf{Totals} &   87,115,458,714 &    803.5\\
\midrule
English One Million\\
\midrule
1 gram &    261,823,186 &   2.6\\
2 gram &    3,383,379,445 & 32.1\\
3 gram &    10,565,828,499 &    94.8\\
4 gram &    12,987,703,773 &    113.1\\
5 gram &    8,747,884,729 & 75.8\\
\textbf{Totals} &   35,946,619,632 &    318.4\\
\midrule
American English\\
\midrule
1 gram &    291,639,822 &   3\\
2 gram &    3,923,370,881 & 38.3\\
3 gram &    12,368,376,963 &    113.9\\
4 gram &    15,118,570,841 &    135\\
5 gram &    10,175,161,944 &    90.2\\
\textbf{Totals} &   41,877,120,451 &    380.4\\
\midrule
British English\\
\midrule
1 gram &    188,660,459 &   1.9\\
2 gram &    2,000,106,933 & 19.1\\
3 gram &    5,186,054,851 & 46.8\\
4 gram &    5,325,077,699 & 46.6\\
5 gram &    3,044,234,000 & 26.4\\
\textbf{Totals} &   15,744,133,942 &    140.8\\
\midrule
English Fiction\\
\midrule
1 gram  &   191,545,012 &   2\\
2 gram  &   2,516,249,717   &   24.3\\
3 gram  &   7,444,565,856   &   68\\
4 gram  &   8,913,702,898   &   79.1\\
5 gram  &   6,282,045,487   &   55.5\\
\textbf{Totals} & 25,348,108,970    &   228.9\\
\midrule
\textbf{Total without 1M}  &    170,084,822,077 &   1553.6\\
\textbf{Total without 1M} &  &  1.53TB\\
\textbf{Total}  & 206,031,441,709 & 1872\\
\textbf{Total}  & & 1.848TB\\
\end{tabular}
\end{table}


\hrulefill

%%%%%%%%%%%%%%%%%%%%%%%%%%%%%%%%%%%%%%%%%%%%%%%%%%%%%%%%

\newday{May 30 2014}

Useful datasets: \url{http://rs.io/2014/05/29/list-of-data-sets.html} In particular Mozilla have released a defect tracking dataset on GitHub\footnote{\url{https://github.com/ansymo/msr2013-bug_dataset}}\citep{Lamkanfi+13}.

Google NGram dataset can be found in Amazon S3\footnote{\url{http://aws.amazon.com/datasets/8172056142375670}}.

\hrulefill

%%%%%%%%%%%%%%%%%%%%%%%%%%%%%%%%%%%%%%%%%%%%%%%%%%%%%%%%

\newday{Jan 20 2013}

\textit{N.B.: The following is a sample entry from Mikhail Klassen's research diary. It is intended to be illustrative of how WriteLaTeX can be used the keep track of your research progress. Some names have been removed from this document for privacy.}

\section*{Initial conditions for the turbulent molecular cloud run}

\subsection*{Inner radius}

The density profile follows an $r^{-3/2}$ power law. To avoid a singularity at the center, an interpolation is done over a radius. This inner radius is defined in the parameter file. It should follow the prescription of a singular isothermal sphere (see Binney \& Tremaine p.305), which is also the definition of the King radius:
\begin{equation}
r_0 \equiv \sqrt{\frac{9\sigma^2}{4\pi G\rho_0}}
\end{equation}
where $\sigma$ is the velocity dispersion and could be estimated as $\sigma = \mathcal{M} c_s$, where $c_s = \sqrt{\gamma P/\rho} = \sqrt{\gamma k_B T / \mu}$ is the sound speed.

The isothermal sound speed in our simulation was estimated
\begin{equation}
c_s = \sqrt{\frac{k_b T}{\mu m_p}}
\end{equation}
I'm unsure why a factor of $\gamma$ was not included. For 30 K, this gives a sound speed of about 34000 cm/s or 0.34 km/s. At a Mach number of 5, this gives a supersonic dispersion of $\sigma$ = 1.7 km/s

This gives an inner radius of $r_0 \approx$ 1.595e17. 

\subsection*{Rotation}

Set the same ratio of rotational to gravitational energy $\beta$ as in Peters et al. 2010a. According to Goodman et al. (1993), this is given by (see equation 6):
\begin{equation}
\beta = \frac{1}{4 \pi G \rho_0} \omega^2
\end{equation}
In practice we can probably use the central density $\rho_c$ instead of determining an average density $\rho_0$. Looking at the numbers from other simulations, we could use an $\omega$ of 1.3e-14.

The link to the Goodman et al. (1993) paper:\\
{\tt http://adsabs.harvard.edu/cgi-bin/bib\_query?1993ApJ...406..528G}

We want to complete our simulation with a similar $\beta$ to check if disks form in the turbulent environment.

The $\omega$ necessary to produce a $\beta = 0.05$ would be
\begin{equation}
\omega = \sqrt{4 \pi G \rho_0 \beta} \approx 7.15\times 10^{-13}
\end{equation}
using $\rho_0 = \rho_c = 1.22\times10^{-17}$.

After testing this, however, I found that the rotation was much too fast. Perhaps using $\rho_0 = \rho_c$ was not a very good assumption at all, since $\rho_c$ is orders of magnitude larger than the average. I wrote a little Python script that sums up all the mass inside the outer radius and divides it by the total volume, defined by the outer radius. In this case, for an outer radius of $5.97402\times 10^{18}$ cm and about 1000 $\Msun$, we get an average density of $2.96415\times 10^{-21}$ g/cm$^3$, which gives us $\omega = 1.114 \times 10^{-14}$.

%\hrulefill

\end{comment}

%%%%%%%%%%%%%%%%%%%%%%%%%%%%%%%%%%%%%%%%%%%%%%%%%%%%%%%%

%\newpage
\bibliographystyle{plain}
\bibliography{LabBook}

\end{document}